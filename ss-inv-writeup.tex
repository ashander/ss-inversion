\documentclass[11pt, reqno]{amsart}
\usepackage[protrusion=true,expansion=true]{microtype}
\usepackage{booktabs}


%\usepackage[ansinew]{inputenc}
%\usepackage{array}
%\usepackage{color}

\usepackage{amsmath}
\usepackage{amsxtra}
\usepackage{amstext}
\usepackage{amssymb}

\usepackage{latexsym}



%%% NEED THIS AT TOP bibliography
\usepackage{natbib}
\citestyle{chicago}
\usepackage{url}

%%% numbers and compressions
%\usepackage{savetrees}
%\usepackage{lineno}

\usepackage{setspace}
\usepackage{graphicx}
\usepackage{subfig}
\usepackage{epstopdf}
\DeclareGraphicsRule{.tif}{png}{.png}{`convert #1 `dirname #1`/`basename #1 .tif`.png}

%\SweaveOpts{prefix.string=figs/evo}

\newcounter{todo}
\setcounter{todo}{1}\newcommand{\todo}[1]{\textcircled{\footnotesize\thetodo}\marginpar{\textcircled{\footnotesize\thetodo}\footnotesize
    Todo: #1}\stepcounter{todo}}




\title{A model for source-sink inversion in great tits}
\author{}
\date{\today}

\begin{document}
\maketitle

Consider a population of great tits in living in two habitat types, deciduous ($D$) and evergreen ($E$) woodlands. 
The population includes individuals that are better adapted to deciduous and to evergreen. 
The habitat types have different areas, which translate to different carrying capacites: $K_D$ for  deciduous and $K_E$ for evergreen.
See Table \ref{T:one} for a summary of notation.

{\em Genetics:} We assume haploid genetics underly adaptation to either deciduous or evergreen woodland. 

{\em Demography:} We assume great tits are semelparous and the offspring produced in each generation are pooled and then randomly re-distributed onto the habitats. 
We also assume an {\em ideal despotic distribution} where the offspring distribute themselves preferentially into ``source'' habitat before filling ``sink'' habitat. 
The definitions of source and sink used in distributing offspring, however, must be relative to the genetics of the population. 
We choose to define the source relative to the average adaptedness of the population, i.e., when the frequency of the $d$ allele in dispersers exceeds 0.5, ($p' > 0.5$) then deciduous habitat is the preferred habitat.

This gives the following dynamics: 
\begin{itemize}
\item Reproduction: the number of offspring of each genotype from reproduction across all habitats
  \begin{subequations}\label{E:repro}
    \begin{align}
      N'_d(t) &= N_{D,d}(t)\lambda_{D,d}+ N_{E,d}(t)\lambda_{E,d}\\
      N'_e(t) &= N_{D,e}(t)\lambda_{D,e}+ N_{E,e}(t)\lambda_{E,e}
      \end{align}
  \end{subequations}
\item Dispersal: the number of offspring distributed to each habitat type in the next generation
  \begin{subequations}\label{E:Ddisp}
    \begin{align}
      N_{D,d}(t+1)&= \frac{ N'_d(t)}{N'_d(t)+N'_e(t)}\begin{cases}K_D \quad \textrm{if } p' > 0.5\\ \min \left[K_D, (N'_d(t)+N'_e(t)-K_E)\right]\end{cases}\\
      N_{D,e}(t+1)&= \frac{ N'_e(t)}{N'_d(t)+N'_e(t)}\begin{cases}K_D \quad \textrm{if } p' > 0.5\\ \min \left[K_D, (N'_d(t)+N'_e(t)-K_E)\right]\end{cases}
     \end{align}
\end{subequations}
  \begin{subequations}\label{E:Edisp}
    \begin{align}
      N_{E,d}(t+1)&= \frac{ N'_d(t)}{N'_d(t)+N'_e(t)}\begin{cases}K_E \quad \textrm{if } p' \le 0.5\\ \min \left[K_E, (N'_d(t)+N'_e(t)-K_D)\right]\end{cases}\\
      N_{E,e}(t+1)&= \frac{ N'_e(t)}{N'_d(t)+N'_e(t)}\begin{cases}K_E \quad \textrm{if } p' \le 0.5\\ \min \left[K_E, (N'_d(t)+N'_e(t)-K_D)\right]\end{cases}
    \end{align}
   \end{subequations}

\end{itemize}


{\em Source} and {\em Sink}: A source is a net exporter, while a sink is a net importer. 
We define ``net import'' and ``export'' by looking at porpotions of dispersers  relative to avialable habitat. 
The ratio of dispersers originating from evergreen habitat to those coming from deciduous habitat is
  \begin{equation}\label{E:rat}
      \frac{ N_{E,d}(t)\lambda_{E,d}+ N_{E,e}(t)\lambda_{E,e}}{N_{D,d}(t)\lambda_{D,d}+N_{D,e}(t)\lambda_{D,e}}.
 \end{equation}
By comparing this ratio to the ratio of habitat avialble, i.e. $K_E/K_D$, we define the inversion criterion:
  \begin{equation}\label{E:invers}
   \eta_{E/D} := \frac{ \frac{ N_{E,d}(t)\lambda_{E,d}+ N_{E,e}(t)\lambda_{E,e}}{N_{D,d}(t)\lambda_{D,d}+N_{D,e}(t)\lambda_{D,e}}}{\frac {K_E}{K_D}}.
 \end{equation}
When inversion criterion $\eta_{D/E} > 1$, deciduous habitat is a source. When less than 1, deciduous habitat is a sink.


\begin{spacing}{1}
\begin{table}
  \renewcommand*\arraystretch{.9}
  \caption{Notation for model.}\label{T:one}
  \centering
  \begin{tabular} {@{}lp{.5\textwidth}l@{}}
    \toprule
    Symbol &Meaning &Units\\     
    \midrule 
    $N_E(t)$ &Total population in $E$vergreen habitat at time $t$ &[number]\\    
    $N_{E,e}(t)$ &Population in $E$ with advantaged allele$e$ at time $t$ &[number]\\    
    $N_{E,d}(t)$ &Population in $E$ with disadvantaged $d$ allele at time $t$ &[number]\\    
   $N_D(t)$ &Total population in $D$eciduous habitat at time $t$ &[number]\\
    $N_{D,e}(t)$ &Population in $D$ with disadvantaged allele$e$ at time $t$ &[number]\\    
    $N_{D,d}(t)$ &Population in $D$ with advantaged $d$ allele at time $t$ &[number]\\    
   $\lambda_{E,e}$ &Geometric growth rate of $E$-advantaged allele $e$&[]\\
   $\lambda_{E,d}$ &Geometric growth rate of $E$-disadvantaged allele $d$&[]\\
   $\lambda_{D,d}$ &Geometric growth rate of $D$-advantaged allele $d$&[]\\
   $\lambda_{D,e}$ &Geometric growth rate of $D$-disadvantaged allele $e$&[]\\
   $K_E$ &Carrying capacity of the Evergreen habitat&[number]\\
   $K_D$ &Carrying capacity of the Deciduous habitat&[number]\\
   $p$      & Frequency of $D$-advantaged allele $d$, i.e., $\frac{N_d}{N_d + N_e}$ &[]\\
    \bottomrule
  \end{tabular}
\end{table}

\end{spacing}
\end{document}  
